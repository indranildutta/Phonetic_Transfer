
The rest of this section is structured as follows: \ref{bengali_english_in_india} provides background information on Bengali and Indian English focusing on vowel systems. \ref{causes} discusses two potential sources of transfer during language processing: global co-activation of multiple languages vs. the local act of switching between languages. Given the highly multilingual setting that characterizes this population, we note that the former is expected to be a constant feature and therefore any observed phonetic interaction is better understood as resulting from the latter. The design of the present study is motivated on the basis of this discussion. A logical consequence of this model is that such phonetic interaction must be highly localized. \ref{duration} elaborates on this by reviewing literature on the duration and `reversibility' of transfer effects. A recurring pattern that emerges from existing research on short-term phonetic transfer is that both languages of the bilingual speaker are not equally affected, although the nature of these reported differences varies greatly across studies. In \ref{asymmetries}, we review some of these findings, focusing on proposed sources of asymmetries, in order to highlight the factors that mediate such phonetic interaction (or fail to do so). We argue that these results point to an urgent need for expanding the range of language pairs and sound categories examined in order to make meaningful generalizations about the nature of short-term phonetic transfer. Following from this, \ref{asymmetry between sounds} discusses reported asymmetries between different sound categories of a single language, and develops the hypotheses of the present study on the basis of these. In \ref{paradigms}, we motivate a secondary aim of this study: to verify the postulated differences between two paradigms for eliciting short-term phonetic interaction in an experimental setup, namely cued picture-naming and code-switching. Finally, we summarize the research questions and hypotheses.



Taken as a whole, these results bear on our original research questions in the following ways: regarding our first question of whether L2 vowels of proficient bilinguals can be affected by short-term transfer from L1 (\textbf{L2 vowels}), our results indicate that this indeed happens during mixed-language use. Both the common category \nt{\ae} and the L2-exclusive category \nt{2} show spectral shifts during mixed-language productions compared to non-switched productions. As hypothesized (c.f. RQ2 and H2, \textbf{Asymmetry}), the results reveal asymmetries in the extent and direction of shift between the two vowels. Specifically, \nt{2} is lowered and fronted, while \nt{\ae} is fronted, with the extent of shift being greater in \nt{2}. Comparison with canonical Bengali vowel productions shows that the shift in \nt{\ae} is towards the corresponding Bengali category \nt{\ae}, while that in \nt{2} is towards the Bengali category \nt{a:}. Regarding our third research question of whether the paradigm used to elicit language switching independently affects patterns of transfer (RQ3: \textbf{Paradigm}), analyzing the role of Task in the results confirms this. We find a greater degree of transfer during code-switching compared to the cued picture-naming task. However, this difference is limited to the F1 dimension, and needs to be replicated. These results and their implications are discussed in the following subsections.


An often discussed issue in transfer research is whether cross-language interaction at the phonetic level during language mixing is triggered by the co-activation of languages that makes representations from both languages available, or by the local act of switching during online processing and articulation of those sounds. 

Studies of short-term phonetic transfer consistently find differences between transfer patterns in L1 vs L2. The bulk of this literature is based on the feature VOT. While vowels have been shown to induce L2-to-L1 influence, the present study demonstrates L1-to-L2 transfer in vowels. 

\subsubsection*{Random effects structure}

Subject and Item were treated as random factors. Given our hypotheses, we included random intercepts for Subject and Item, by-subject random slopes for Vowel (formant targets for vowel categories differ across individuals) and Context (individuals differ in their response to the experimental condition), and a by-item random slope for Context (the effect of experimental conditions differ across words). 

We tested the significance of our hypothesized explanatory variables using t-values and p-values for individual variables in the models, and tested overall model fit by comparing the full model to a corresponding null model that lacks the variable of interest using ANOVAs. The following subsections summarize and discuss the model outputs for F1 and F2 respectively.


We extracted formant estimates from five uniformly spaced time points in the vowel: from the time at 5\% into the vowel (at the start of articulation) to 45\% (at the mid-part, or steady state, which is expected to correspond most closely to the articulatory target for the category). As expected, w


The significant Time*Vowel interaction shows that the F1 trajectory differs between the vowels as expected, since the two categories have different acoustic targets.  

Vowel is not a significant predictor of F1. This is initially surprising, given that the phonological category is expected to systematically affect vowel quality. However, a

Given that estimates for fixed effects are pooled across participants, this explains the lack of reliable differences between \nt{2} and \nt{\ae}.

This indicates a uniform scaling of formant trajectories in the mixed condition, suggesting that the underlying acoustic targets for these vowels are different during non-switched vs mixed utterances. 



\subsubsection*{Model output summary}

First, to test our first and second research questions (\bf{L2 vowels and Asymmetry}), we fitted a model with vowel*time, and vowel*context as fixed effects. The transfer term (vowel*context) is significant in this model (model coefficient/ unstandardized effect size ($\beta$)= -2.330e-01, standard error (SE)=  5.244e-02, t= -4.44, p=0.0002). Comparing this to a null model that lacks the transfer term shows that the full model is significantly better ($\chi^2(2) = 14.42, p=  0.0007$). To confirm whether the effect of context is moderated by vowel identity, we compared a second null model that contained Context as a fixed effect, but lacked the Context*Vowel interaction. The full model was better ($\chi^2(1) = 12.51, p= 0.0004$).

To examine the third research question, \textbf{Paradigm}, we fitted a model with vowel*time and task*vowel*context as fixed effects. This three-way interaction term is significant ($\beta$= -1.955e-01, SE=  5.453e-02, t= -3.59, p= 0.0003). We also find a significant effect of task on FFs ($\beta$=3.300e-01, SE=2.668e-02, t=12.371, p=2e-16). Comparing this model to a null model that lacks the fixed effect Task ($\chi^2(4) = 572.61, p=  2.2e-16$) shows that the full model is better. Because Task has an independent effect on F1, and adding interactions increases model parameters, we fitted a second null model with Task as a separate fixed effect without the interaction term to test whether the interaction term is needed. However, the full model was still a better fit for the data ($\chi^2(3) = 23.82, p=  2.722e-05$). Next, we tested two other null models to verify that the full model with the three-way interaction term is indeed the optimal model for the data, by replacing the three-way interaction term with: (i) vowel*context + task*context (to test if task only interacts with context, rather than the transfer term vowel*context). The full model is better ($\chi^2(2) = 23.28, p= 8.78e-06$); (ii) vowel*context + task*vowel (to test if task only interacts with vowel, rather than the transfer term vowel*context). The full model is better ($\chi^2(2) = 13.36, p= 0.001$). This shows that the three-way interaction is the best fit for the data. Because an independent effect of task on FFs is unexpected (c.f. model interpretation below), we wanted to examine if the effect might be local, i.e. decrease over time. We added a task*time interaction term to the full model, which is significant ($\beta$= -1.216e-01, SE=  1.347e-02, t= -9.028, p= 2e-16 ) and improves model fit ($\chi^2(1) = 81.22, p= 2.2e-16$).

Next, to examine the dynamics of transfer, we added a three-way interaction between the transfer term (vowel*context) and time. This term has a marginally significant effect ($\beta$= -4.604e-02, SE=  2.692e-02, t= -1.71, p= 0.08), but does not significantly improve model fit compared to a null model lacking the vowel*context*time term ($\chi^2(2) = 3.41, p= 0.18$).

Thus, the optimal model was: f1 $\sim$ time*vowel + vowel*context*task + time*task + (1 + context|word) + (1 + context + vowel|subject). Table \ref{table_f1} summarizes the model coefficient estimates ($\beta$), standard errors (S.E.), t-values, and p-values for this model. p-vales smaller than 0.05 are starred. 



\subsubsection*{Model output summary}

First, to test RQ1 (\textbf{L2 transfer}) and RQ2(\textbf{Asymmetry}), we fitted a model with vowel*time, and vowel*context as fixed effects. The transfer term (vowel*context) is significant in this model (standardized effect size ($\beta$)= 1.234e-01, standardized error (SE)=  2.055e-02, t= 6.00, p=5.46e-06). Comparing this to a null model that lacks the transfer term shows that the full model is significantly better ($\chi^2(2) = 16.20, p= 0.0003$). To confirm whether the effect of context is moderated by vowel identity, we compared a second null model that contained Context as a fixed effect, but lacked the Context*Vowel interaction. The full model was better ($\chi^2(1) = 16.02, p= 6.245e-05$).

To test RQ3 (\textbf{Paradigm}), we fitted a model with vowel*time and task*vowel*context as fixed effects. While task has an independent effect on F2 ($\beta$= -8.758e-02, SE= 1.756e-02, t= -4.98, p= 6.18e-07), the three-way interaction term is not significant ($\beta$= 3.349e-02, SE= 3.565e-02, t= 0.94, p= 0.34). However, there are significant two-way interactions between task*context ($\beta$= -1.099e-01, SE= 2.483e-02, t= -4.427, p= 9.66e-06) and task*vowel ($\beta$= -8.983e-02, SE= 2.535e-02, t= -3.544, p= 0.000396). Comparing this to a null model that lacks the fixed effect Task confirms that the full model is still a better fit for the data ($\chi^2(4) = 572.61, p=  2.2e-16$), indicating that an optimal model should contain the term Task. To confirm whether Task interacts with any of the other terms, we fitted a model with Task as a separate fixed effect without the interaction term. Comparing this to the full (interaction) model shows that the latter is still a better fit ($\chi^2(3) = 44.8, p= 1.02e-09$). Next, we tested two other null models to verify that the three-way interaction term is indeed optimal for the data, by replacing the three-way interaction term with: (i) vowel*context + task*context (to test if task only interacts with context, rather than the transfer term vowel*context). The full model is better ($\chi^2(2) = 17.28, p= 0.0001$); (ii) vowel*context + task*vowel (to test if task only interacts with vowel, rather than the transfer term vowel*context). The full model is better ($\chi^2(2) = 28.45, p= 6.633e-07$). This shows that the three-way interaction is the best fit for the data.

Because an independent effect of task on FFs is unexpected (c.f. model interpretation below), we wanted to see if the effect might be local, i.e. decrease over time. We added a task*time interaction term to the full model, which is significant ($\beta$= 6.237e-02, SE=  8.876e-03, t= 7.027, p= 2.23e-12 ) and improves model fit ($\chi^2(1) = 49.271, p= 2.23e-12$).

Next, to examine the dynamics of transfer, we added a three-way interaction between the transfer term (vowel*context) and time. This term is not significant ($\beta$= -6.274e-03, SE= 1.774e-02, t= -0.35, p= 0.72), and the addition does not significantly improve model fit compared to a null model lacking the vowel*context*time term ($\chi^2(2) = 2.2443, p= 0.32$).

Thus, the optimal model is: f2 $\sim$ vowel*time + vowel*context*task + task*time + (1+context|word) + (1+context+vowel|subject). Table \ref{table_f2} summarizes the model coefficient estimates ($\beta$), standard errors (S.E.), t-values, and p-values for this model. p-values smaller than 0.05 are in starred.


For the present study, we focus on a variety of IE spoken by an educated elite, used in news channels, often taught as a target in classrooms, and generally considered to be free of any particular regional affiliation \citep[c.f.][who reports this variety to be ``a \textit{de-facto} norm", and thus calls it General(ized) Indian English]{masica1972sound}. This sound system is described in section \ref{vowel systems}. However, recent research comparing the segmental and suprasegmental properties of IE spoken in different parts of the country also report many commonalities (see \cite{sirsa2013effects} for a review). It has thus been suggested that in spite of the varied L1s of its speakers, IE has a target phonology that is distinct from any of these, as well as from other native varieties of English. Thus, it is fruitful to think of regional variations as resulting from L1-influence on a common underlying target. Excepting a small minority of specific communities \citep{pandey201517, wells1982accents, coelho1997anglo}, there are very few L1 speakers of IE. Thus, English is largely learned and used as an L2.


Specifically, a pattern that emerges from the body of work on VOT is that a shift in long-lag VOT towards the short-lag norms of the other language is much more consistent and systematic than the converse \citep{tobin2017phonetic, olson2016role,bullock2009trying,antoniou2011inter,chang2012rapid}. In general, short-lag VOTs seem to resist accommodative shift. \cite{bullock2009trying} suggest that because long-lag languages offer a greater range of acceptable VOTs, there is more `room' for movement. In contrast, shift in short-lag languages could risk the loss of a phonological contrast (VOT being the primary cue for voicing contrast in these languages).  This is a language-specific phonological constraint on transfer.
In a similar vein, \cite{antoniou2011inter} observe that while the VOT of English stops shifted towards Greek (a short-lag language), the reverse was seen only in a specific subset of sounds: word-medial non-nasalized stops. They attribute this to the low frequency of these sounds in Greek, making them less ``stable", and thus more amenable to cross-language influence. This suggests that the distribution of sounds in a language could affect the kind of transfer patterns observed.


Since Bengali has a four-way laryngeal contrast, VOT does not encode a binary voiced-voiceless distinction. Moreover, VOT is not the only relevant cue for the corresponding phonological categories in the language \citep{dmitrieva2020acoustic}. This means that any effect of L1 Bengali on L2 English VOT is likely to be affected by additional phonological factors. Thus, avoiding any potential confounds was another reason we chose to focus on vowel quality as a target for transfer in the present study. 

Research comparing bilingual speakers' phonological systems to monolinguals treats them as relatively stable over time. The difference from monolingual norms is interpreted as the cumulative result of cross-language influence over long periods \citep[e.g.][]{guion2003vowel, caramazza1973acquisition}. 

over time, increasing exposure to an L2 can reduce effects of L1 influence, leading to a more fine-grained separation and native-like production of foreign contrasts. For example, \cite[][vowel quality]{bohn1992production} 
observed that approximately 7 years' difference in L2 exposure between groups having the same age of acquisition led to a significant difference in the amount of cross-language influence. 

F2 model interpretation:
As with F1, we find that Time significantly influences F2 values in the model, reflecting the dynamic nature of speech which causes acoustic variation over the course of articulation. Moreover, the Time*Vowel interaction shows that this trajectory differs across the two vowels: while the positive slope for Time indicates that F2 increases i.e. vowels are fronted with time, the negative slope of the Time*Vowel interaction shows that this effect is smaller for \nt{2}. This is expected, as figure \ref{vowels_unilingual_ie_bengali} confirms that the category \nt{\ae} is overall more fronted than \nt{2}. 

Unlike in the F1 dimension, Vowel is a significant predictor of F2, with \nt{2} having significantly lower F2, i.e. being produced as more backed, than \nt{\ae}. As with vowel height, there is no significant independent effect of Context, showing that mixed language use does not lead to uniform fronting or backing across vowels. This is not surprising, as we expect movement to be towards the corresponding L1 category, and thus differ across vowels. The negative slope of the estimate indicates that on average, vowels have lower F2 i.e. are backed, in the nonswitch (e) condition. In other words, vowels are fronted in the mixed condition. This effect is greater in \nt{\ae}, as indicated by the negative slope of the Vowel*Context interaction. In the F2 dimension, too, the lack of a significant Vowel*Context*Time interaction shows that transfer patterns do not change across time, suggesting that the underlying acoustic targets are different in the two conditions.

We see a significant effect of Task on F2, such that vowels in the code-switching (s) task have lower F2, i.e. are more backed than those in the picture-naming (p) task. As with F1, this is an unexpected result; we have no principled reason to expect code-switching to induce an across-board backing of vowels. Vowels are backed in the code-switching task, independent of experimental condition. A significant interaction with Time, with a positive slope for the estimate, shows that this effect is local: it is strongest at the beginning of the vowel, and reduces over the course of the vowel. This aligns with the possibility that the unexpected Task effect is an artifact of the paradigm and the particular stimuli used. The Vowel*Context*Task interaction is of interest, as it relates to our research question of how the test paradigm affects transfer. Unlike vowel height, we do not find patterns of shift in vowel backness (F2) to differ across paradigms, as indicated by the lack of a significant effect for the fixed effect. 


The literature on short-term phonetic transfer has frequently noted differences in transfer patterns between different sound categories in a language. This has been attributed to both general linguistic principles such as contrast preservation, and language-specific factors such as frequency and distribution of categories. Examining these asymmetries more closely, and understanding their sources, helps us figure out what factors affect/moderate transfer and thus, by extension, what level of representation and processing this interaction happens at, and the nature of short-term intra-speaker variation. 
We examined transfer in two L2 categories, \nt{2} and \nt{\ae}, and hypothesized an asymmetrical shift between the two vowels owing to  differences in their position in the
\begin{itemize}
	\item their position in the Indian English vowel space: \nt{\ae} is in a more crowded part of the vowel space
	\item the presence/absence of a corresponding category in the L1 Bengali inventory: \nt{\ae} is a common phonological category in Bengali and English, while \nt{2} does not exist as a category in Bengali
\end{itemize}


\cite{olson2013bilingual} proposes that in both scripted and spontaneous code-switching paradigms, ``discursive properties may partially drive the production of code-switches, masking underlying effects of interaction", and thus proposes cued picture-naming as a preferable alternative. If such discursive properties independently affect the phonetic outcome of language mixing in an experimental setup, this calls to question the comparability of results across studies using these different paradigms. 