% !TeX document-id = {6d544fe2-4255-4108-8022-d809ff1d2c53}
% TeX program = xelatexmk %doesn't work, no xelatexmk option under 'build'. Changed default engine to xelatex (options-->configure-->build-->default build-->xelatex)
% see http://info.semprag.org/basics for a full description of this template
%\documentclass[times,linguex]{glossa}
\documentclass[12 pt]{article}
\usepackage[letterpaper]{geometry}
\usepackage{times}
\geometry{top=1.5in, bottom=1.0in, left=1.0in, right=1.0in}

\usepackage[hidelinks]{hyperref}
%\renewcommand{\sectionautorefname}{\S}
\usepackage{xcolor}
\hypersetup{
	colorlinks,
	linkcolor={blue!50!black},
	citecolor={blue!50!black},
	urlcolor={blue!80!black}
}


\usepackage{graphicx} %for images
\graphicspath{{images/}}
\usepackage{caption}
\captionsetup[table]{skip=0pt,singlelinecheck=off}
\captionsetup[figure]{skip=0pt,singlelinecheck=off}
\usepackage{subcaption}

\usepackage{standalone} %to include separate .tex files that contain mages

\usepackage{booktabs}
\usepackage{enumitem} %customize list numberings

\usepackage[natbibapa]{apacite}
\usepackage{natbib}

\usepackage{tipa}
\newcommand{\nt}[1]{\textipa{[#1]}} % narrow transcription
\newcommand{\wt}[1]{\textipa{/#1/}} % wide transcription


\usepackage{fancyhdr}
\pagestyle{fancy}
\fancyhf{}
\renewcommand{\headrulewidth}{0pt}
\fancyhead[R]{\thepage}

\usepackage{setspace}
\doublespacing

\usepackage{comment}

\newlength\mystoreparindent
\newenvironment{myparindent}[1]{%
	\setlength{\mystoreparindent}{\the\parindent}
	\setlength{\parindent}{#1}
}{%
	\setlength{\parindent}{\mystoreparindent}
}

\usepackage{easyReview}

%\usepackage[noae]{Sweave} %to get bold text?

\title{Mixed language processing increases cross-language phonetic transfer in Bengali-English bilinguals}



% \author[Auromita \& Indranil]% short form of the author names for the running header. If no short author is given, no authors print in the headers.
% {%as many authors as you like, each separated by \AND.
	%   \spauthor{Auromita Mitra\\ 
		%   \institute{The EFL University}\\
		%   \small{%105, Bd. Raspail, 75005 Paris\\
			%   auromita.mitra@gmail.com}
		%   }
	%   \AND
	%   \spauthor{Indranil Dutta \\
		%   \institute{Jadavpur University}\\
		%   \small{%Warmoesberg 26, 1000 Brussel\\
			%   indranildutta.lnl@jadavpuruniversity.in}
		%   }%
	% }
%\usepackage{gb4e}

\begin{document}
	\bibliographystyle{apacite}
	%\sffamily
	%\maketitle

\subsection{Participants} \label{participants}
Participants were recruited through an open call at the EFL University campus in Hyderabad. Interested volunteers completed a brief language background questionnaire which was used to screen participants as described below. 

As noted in section \ref{bengali_english_in_india}, India has an extremely multilingual an linguistically diverse population. The university campus comprises of students from different parts of the country and world, with varying L1s and different varieties of English. Our aim in screening participants was to minimize between-participant random variability, while maintaining ecological validity considering the actual day-to-day settings in which transfer behaviors likely take place. We chose two metrics to do this: (i) where the participant grew up; and (ii) amount of formal education in Bengali. The first ensures that the variety of Bengali that individuals are exposed to in their surroundings is similar across participants. This would mean that the target representations for Bengali vowels would be expected to be similar. This was especially important because we were not eliciting unilingual L1 Bengali data in this study, and were using an existing data source to estimate the position of Bengali vowels in the formant space. Based on previous literature (c.f. section \ref{asymmetry between sounds}), we expect that L2 vowels shift towards related L1 categories during mixed language utterances. Thus, we wanted to ensure that any between-participant asymmetries in the pattern of shift reflected individual differences in transfer, rather than different acoustic targets. Another reason to control for location was that it is a fairly reliable predictor for language(s) of education, owing to how the education system functions in the country (see Supplementary materials for details).

The second metric is Medium of Instruction (MoI) in school. Most schools in India follow the Three Language Policy, wherein formal education is provided in English, Hindi, and a regional language. \alert{(The idea and implementation of this policy, and its implications for linguistic minorities, has been a matter of much public debate as well as scholarly work; for discussions see, e.g. \cite{tollefson2014language, jhingran2009hundreds, khubchandani1997language, mohanty2009multilingual, ramanathan2005rethinking}).} Sustained formal education in a language is a reliable predictor of LSRW (listening, speaking, reading, writing) skills. Since this study uses written stimuli, we wanted to ensure that all participants were sufficiently comfortable with reading and speaking both Bengali and English, to avoid variability due to reading/task effort.  

Given these considerations, we only recruited participants who had grown up in the state of West Bengal and lived there for the majority of their lives, had parents who both spoke Bengali, and had received at least 5 years of formal education in Bengali (all respondents had received 5+ years of education in English and were pursuing university degrees that are taught in English), and self-reported being proficient in speaking, understanding, and reading Bengali. 

A total of 28 bilingual speakers of Bengali and English (18 female) responded to the recruitment call. Using the inclusion criteria reported above, 10 volunteers (5 female, 5 male, age range 19 to 28) were invited to participate in a speech production study, and were compensated for their time. All participants reported normal hearing, and normal or corrected-to-normal vision. After completion of the study, we followed up with a detailed language background survey at a later date to learn more about the language profiles of the participants. This survey was a modified and consolidated version of three popular bilingualism profiling tools: the Bilingual Language Profile (BLP) \citep{blp}, the Language Experience and Proficiency Questionnaire (LEAP-Q) \citep{leap-q} which incorporates language history and self-assessed proficiency, and the Bilingual Switching Questionnaire (BSWQ) \citep{language_switching_questionnaire}. Together, these incorporate questions about language history, usage, attitudes, and switching experience. \alert{table ... summarizes the key information. The complete questionnaire, along with responses, is available in the ``Data" component of the OSF repository for this project (c.f. section \ref{data_availability}). --footnotes?}

In addition to Bengali and English, all participants reported knowing other languages, the most common being Hindi. Since this situation is representative of the present population and our study concerns only Bengali-English interaction, we only focused on these languages in the survey.
The average age of acquisition was 2.6 for English (range 0 i.e. since birth to 6) and 0.25 (range 0 to 2) for Bengali. Thus, all participants can be described as simultaneous bilinguals, having acquired both English and Bengali at a young age. They report being comfortable using both languages at a young age, with the average being 7.5 for English and 0.3 for Bengali (on a likert scale rating where 0=for as long as I can remember, 5=since I was 5 years old, and so on). As mentioned earlier, all participants had attended schools where Bengali and English were both used as Medium of Instruction (MoI). On average, participants had had more than 16 years (range 7 to 20+) of formal education in English and 11.8 years (range 8 to 15) in Bengali at the time of the study. As described in the inclusion criteria, all participants reported having spent over 15 years in a Bengali-speaking region. At the time of recording, they were living in or around the EFL University campus in Hyderabad, which has a linguistically diverse population. In terms of usage, participants show a range of language usage patterns. Almost 100\% of the participants report exclusively using Bengali with little to no English while interacting with family, and English at school/the workplace. However, language use with friends varies greatly, likely reflecting the language composition of the friend circle. 
The questionnaire asked participants to rate their proficiency in each language. All participants rated themselves as highly proficient in speaking and understanding both Bengali and English. While self-ratings for reading and writing English were at ceiling, ratings for Bengali literacy (reading and writing) showed some variation, with an average of 4.87 (on a 0-6 likert scale, range 3 to 6) for reading and 4.25 (range 3 to 6) for writing. This is a common situation for many educated sections of society, since institutions of higher education beyond the school level largely use English (for discussions about the language politics of higher education in India, see, for example, \cite{mohanty2009multilingual}). 
The language attitude section of the questionnaire reveals interesting insights about the linguistic landscape of the population, and how people experience language in such a setting. When asked how much ``like themselves" participants feel while using each language, response were at ceiling for both languages. Similarly, most participants identified equally strongly with both an ``English-speaking" and a ``Bengali-speaking" culture, possibly because for most people, such linguistic-cultural identities are not clearly delineated and not exclusive of one another. 
These responses shed light on how multiple languages fit seamlessly into different facets of daily life to create a ``continuous" linguistic experience (c.f. ``plurality" as defined by \cite{khubchandani2021plural}). In such a scenario, it is not surprising that language-switching is common, expected, as well as used in creative ways with communicative purpose. Responses to the section on language switching demonstrate this. On average, participants do not report being often unable to recall words in the target language while speaking Bengali (average rating 2.5 on a 1-6 likert scale where 1=never and 6=always) or English (average rating 2.3), indicating that switching is not strictly necessary for filling lexical gaps. Indeed, participants do not report switching between languages without awareness (1.8) or control (1.25). In spite of this, participants report a proclivity to code-switch (rating 3.5). Importantly, a large number of participants report volition while switching languages (4.12), and many report this to be situationally motivated (3.12). These responses indicate that participants are highly aware of their switching habits and move between languages for communicative purposes. This fact, combined with the linguistic situation described here and in section \ref{bengali_english_in_india}, are important to bear in mind while interpreting the results of a transfer study in the present population. 

Given our inclusion criteria, we were able to recruit 10 participants for this study. Although existing studies in the field have used comparable participant numbers, we acknowledge that this is a small sample size. As detailed above, even after a recruitment call that already controlled for language background and education level (recruiting Bengali-English bilinguals in a university campus), we were able to include only 10 out of 28 volunteers (35.7\%) in the final study. Note that the criteria used here establish only a minimal level of between-participant similarity in exposure, dominance, habits, and proficiency, such as is extremely common in most comparable studies. This demonstrates the degree of linguistic variability in the present population. Norms for experimental research on bilingualism have largely been established in a Western context. In settings where linguistic variability at the population level is much greater than these canonically studied groups, establishing comparable levels of experimental control requires significantly greater resources. Given that small sample sizes are likely to have low statistical power, replication with larger sample sizes are important to confirm the observed effects, and provide much-needed data diversity (c.f. discussion in section \ref{introduction}). A more detailed discussion of power analyses in bilingualism research is included as supplementary material on OSF (c.f. section \ref{data_availability}). We hope that this can facilitate more conversation around establishing norms for power and sample size in our field.  



\bibliography{references}

\end{document}